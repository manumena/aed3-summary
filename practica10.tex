\section{Práctica 10}

\setcounter{subsection}{1}
\subsection{}
Puede modelarse como un problema de grafos donde cada nodo es una tarea y existe una arista entre cada par de vértices que comparte un recurso.

\subsection{}
\subsubsection{}
Verdadero. Si $H$ es un subgrafo de $G \Longrightarrow \chi(H) \leq \chi(G)$. Como $\chi(K_s) = s$ y $K_s$ es subgrafo de $G$, entonces $\chi(G) \leq s$.

\subsubsection{}
Falso, por ejemplo, $C_5$.

\subsubsection{}
Falso. El subgrafo máximo completo de $C_5$ es $K_2$ mientras que su número cromático es 3. El número cromático de $K_3$ tambíen es 3.

\subsubsection{}
Falso. Puedo tener un grafo donde un vértice y sólo él se une el resto de los vértices. Su grado puede ser tan grande como quiera pero su número cromático es 2.

\subsection{}
Se agrega un nodo al grafo por cada curso en donde dos cursos comparten una arista si tienen algun inscripto en común. A partir de este modelo se colorea el grafo en donde cada color representa un horario distinto.

\subsection{}
Se arma un grafo a partir de la matriz de distancias en donde cada nodo es una torre y dos nodos comparten una arista entre sí en caso de tener una distancia mayor o igual a 50. El número cromático de este grafo es el menor número de frecuencias distintas.