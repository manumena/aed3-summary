\section{Práctica 11}

\setcounter{subsection}{1}
\subsection{}
\subsubsection{}
Verdadero. $Correspondencia = \emptyset$

\subsubsection{}
Falso. $K_1$ no tiene recubrimento de vértices por aristas ya que al no tener
aristas no se puede cubrir el vértice.

\subsubsection{}
Verdadero. Un nodo asilado.

\subsubsection{}
Verdadero. Todos los vértices

\subsubsection{}
Verdadero. Cualquier correspondencia tiene como máximo $\frac{n}{2}$ ya que
cada arista se corresponde con exactamente 2 vértices debido a que no pueden
haber 2 aristas que compartan algún vértice. Cualquier recubrimiento de
vértices por aristas tiene como mínimo $\frac{n}{2}$, ya que la mínima
cantidad de aristas que puede lograrse es cuando no se repiten nodos, por lo
que por cada arista se tocan 2 nodos. Entonces cualquier correspondencia
$\leq \frac{n}{2} \leq$ cualquier cubrimiento de vértices por aristas.

\subsection{}
\subsubsection{}
Un grafo tiene un recubrimiento de vértices por aristas si y sólo si no
tiene vértices aislados.

\begin{proof}

	~

	$\Longrightarrow )$

	Supongo que un grafo tiene algún vértice asilado. Entonces no hay manera
	de formar un recubrimiento de vértices por aristas ya que ese nodo no
	puede ser cubierto por no tener aristas que incidan sobre él. Absurdo.

	~

	$\Longleftarrow )$

	El grafo no tiene vértices aislados, entonces puedo formar un
	recubimiento donde por cada nodo pongo alguna de sus aristas.
\end{proof}

\subsubsection{}

Por inducción:

Caso base: $n = 2$

Cualquier recubrimiento de aristas tiene 1 arista.

Hipótesis inductiva: Si un grafo tiene $n$ vértices, cualquier recubrimiento
de vértices por aristas tiene al menos $l{n/2}l$ aristas.

Paso inductivo:

Sea un grafo $G$ de $n + 1$ vértices, quiero ver que cualquier recubrimiento
de aristas por vértices tiene al menos $l(n + 1)/2l$ aristas. Es decir que si
$n$ es par tiene al menos $n/2$ y si es impar, $(n + 1)/2$.

Elijo cualquier vértice $v$ de $G$ y lo saco. Como $G - v$ tiene $n$ vértices
, sé por hipótesis inductiva que $G - v$ tiene un recubrimiento de vértices
por aristas de al menos $ln/2l$ aristas. Entonces si ahora agrego alguna de
las aristas de $v$ al recubrimiento, si $n$ es par, tiene al menos
$n/2 + 1 = (n + 2)/2 \geq (n + 1)/2$ aristas, y si $n$ es impar, tiene al
menos $(n + 1)/2 + 1 = (n + 3)/2 \geq (n + 2)/2$ aristas.

\setcounter{subsubsection}{3}
\subsubsection{}
Supongamos que un recubrimiento minimal contiene un circuito. Podría
eliminarse una de las aristas del circuito y todos los vértices permanecerían
recubiertos, incluso los que son incididos por la arista eliminada, que están
cubiertos por los extremos del camino que se forma. Pero entonces el
recubrimiento no era minimal. Absurdo.

\subsubsection{}
$ln/2l$

\subsubsection{}
\begin{lema}
	Un recubimiento es minimal si y sólo si no contiene caminos ni circuitos
	simples de longitud mayor o igual a 3.
\end{lema}

\begin{proof}

	~

	$\Longrightarrow )$

	Si el recubrimiento contiene un camino o circuito simple de longitud
	mayor o igual a 3, entonces puedo eliminar una arista cualquiera del
	circuito, o una del camino que no sea de los extremos, y los nodos sobre
	los que incide esta arista permanecerían cubiertos por las aristas
	adyacentes a esta. Entonces no sería un recubrimiento minimal.

	~

	$\Longleftarrow )$

	El recubrimiento no contiene caminos ni circuitos simples de longitud
	mayor o igual a 3. Para cualquier arista que tome, si la saco, es seguro
	que alguno de los nodos de los extremos quedaría sin cubrimiento, ya que
	de lo contrario, esa arista pertenecería a un camino o circuito simple de
	longitud mayor o igual a 3.


\end{proof}

\setcounter{subsection}{4}
\subsection{}
$G = (V, E)$ bipartito con $n = |V|$ y $m = |E|$.

$\alpha =$ cantidad de nodos de conjunto independiente máximo.

$\beta =$ cantidad de nodos de cubrimiento de aristas por vértices cualquiera.

$m \leq \alpha \times \beta$

\begin{proof}
	$E_i = \{e \in E / e$ toca al vértice $i\}$

	$E = \cup_{i = 1}^{\beta}E_{v_i}$ por ser cubrimiento de aristas por
	vértices.

	$\{v_1, ..., v_{\beta}\}$ es cubrimiento de aristas por vértices.

	$m = |E| = |\cup_{i = 1}^{\beta}E_{v_i}| \leq \sum_{i = 1}^{\beta}|E_{v_i}| \leq \sum_{i = 1}^{\beta}d(v_i)$ ($d(v_i) \leq \alpha$) $\leq \sum_{i = 1}^{\beta}\alpha = \alpha \times \beta$

	Voy a probar que $d(v_i) \leq \alpha$. Como $G$ es bipartito se que tiene
	dos conjuntos independientes $a \leq \alpha$ y $b \leq \alpha$. Entonces
	cada nodo perteneciente a alguno de estos dos conjuntos tiene como mucho
	$\alpha$ vecinos.
\end{proof}

\setcounter{subsection}{8}
\subsection{}

\setcounter{subsubsection}{1}
\subsubsection{}

$G = (V_1 \cup V_2, E)$ bipartito tiene una correspondesncia completa de $V_1$ en $V_2 \Longrightarrow \forall W \subseteq V_1$ se cumple $|\Gamma(W)| \geq |W|$

\begin{proof}

	~

	$\Longrightarrow )$

	$G$ tiene una correspondencia completa de $V_1$ en $V_2$, entonces existe
	un correspondencia $C$ tal que todo vértice de $V_1$ es incidente a un
	arco de $C$. Es decir que cualquier vértice de $V_1$ que tome es adyacente
	al menos a algún vértice de $V_2$. Por lo tanto cualquier subconjunto de
	$r$ vértices de $V_1$, tiene al menos $r$ vértices adyacentes a $V_2$, al
	menos uno por cada vértice de $V_1$.

	~

	$\Longleftarrow )$

	Quiero ver que $\forall W \subseteq V_1$ se cumple $|\Gamma(W)| \geq |W| \Longrightarrow G$ tiene una correspondencia de $V_1$ en $V_2$.

	$G$ tiene una correspondencia completa de $V_1$ en $V_2 \Longrightarrow$

	$G$ tiene uan correspondencia de $|V_1|$ ejes $\Longrightarrow$

	$G$ tiene una correspondencia máxima de $|V_1|$ ejes $\Longrightarrow$

	$G_{st}$ tiene un flujo máximo de valor $|V_1| \Longrightarrow$

	$G_{st}$ tiene un corte mínimo de capacidad $\Longrightarrow$

	En $G{st}$ existe un corte de capacidad $|V_1|$ y todo corte tiene
	capacidad $\geq |V_1| \Longrightarrow$

	En $G{st}$ \begin{enumerate}
		\item Existe un corte que corta $|V_1|$ ejes
		\item Todo corte corta al menos $|V_1|$ ejes
	\end{enumerate}

	Para demostrar que $G$ tiene una correspondencia completa de $V_1$ en
	$V_2$ alcanza con demostrar $\forall W \subseteq V_1$, $|\Gamma(W)| \geq |W| \Longrightarrow 1)$ y $2)$.

	\begin{enumerate}
		\item Por ejemplo $S = \{s\}$.
		\item {
			Sea $S$ cualquier corte en $G{st}$. Defino $T = V(G{st}) - S$.

			\begin{enumerate}[label=\alph*)]
				\item $\forall v \exists V_1 \cap T$ se corta $(s, v)$
				\item $\forall v \exists S \cap \Gamma(V_1 \cap S)$ se corta $(v, t)$
				\item $\forall v \exists T \cap \Gamma(V_1 \cup S) \exists V_1 \cup S$ tal que se corta $(w, v)$
			\end{enumerate}

			En total se cortal al menos esta cantidad de ejes: $|V_1 \cup T| + |S \cup \Gamma(V_1 \cup S)| + |T \cup \Gamma(V_1 \cup S)| = |V_1 \cup T| + |\Gamma(V_1 \cup S)|(\geq |V_1 \cup S|) \geq |V_1|$.
		}
	\end{enumerate}
\end{proof}

\setcounter{subsection}{14}
\subsection{}
Se puede usar el método \textit{Ford y Fulkerson} recorriendo la red
residual pero en lugar de aumentar el flujo, disminuirlo, incrementando los
caminos de aumento de la red residual.
flujo