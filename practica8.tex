\section{Práctica 8}

\setcounter{subsection}{3}
\subsection{}

$\Longrightarrow$ )

$G$ tiene un circuito euleriano, osea que si parto de un nodo $v_0$ puedo volver al mismo pasando por todas las aristas, sin repetir alguna. A su vez, todos los nodos de $G$ tienen grado par.

Puede ocurrir que $G$ sea un circuito simple, en cuyo caso tomo $G$ como la única partición, o puede que no. En ese caso puedo formar un circuito simple $C_1$ con un conjunto de aristas de $G$, ya que sé que al haber un circuito hay al menos un circuito simple.

Las aristas de $C_1$ forman un conjunto de la partición que busco. Sea $G_1 = G \setminus C_1$. Todos los vértices de $G_1$ siguen teniendo grado par. Si $G_1$ no tiene aristas ya tenemos la partición que buscábamos. Sino, seguro $G_1$ tiene un circuito simple $C_2$, por ser euleriano. Así iteramos hasta quedarnos sin aristas y nos queda la partición $\{C_1, ..., C_k\}$. \\

$\Longleftarrow$ )

$G$ puede particionarse en circuitos simples $\{C_1, ..., C_k\}$. Considero $G_0$ el grafo con los mismos nodos que $G$ pero sin aristas, sólo nodos aislados. Todos tienen grado 0, osea que todos son de grado par. Ahora agrego $C_1$. Como $C_1$ es ciclo simple, todos sus nodos tienen grado 2, por lo que agregar $C_1$ a $G_0$ forma un grafo $G_1$ en donde cada nodo aumenta de grado en 2 o permanece igual, por lo que son pares. Itero hasta agregar todos los circuitos.

El grafo que se forma es $G$, es conexo y todos sus nodos son de grado par, por lo que $G$ es un grafo euleriano.

\subsection{}

$G$ es un grafo conexo con todos sus vértices de grado par, por lo tanto es un grafo euleriano. Entonces $G$ se puede particionar en ciclos simples.

Para cualquier vértice $v \in G$ se sabe que todas sus aristas pertenecen a algún ciclo simple. Por cada arista de $v$ hay otra que también sale de $v$ que pertenece al mismo ciclo simple. Es decir que $v$ conecta $\frac{d(v)}{2}$ ciclos simples, ya que puedo elegir pares de aristas de $v$ que pertenecen al mismo ciclo.

Al sacar uno de esos pares de aristas, es posible que se forme otra componente conexa porque ese par pertenece a un ciclo simple $C$, lo que hace que se forme un camino simple ya que las aristas que elimino son adyacentes. Para formarse otra componente conexa debe ser $v$ el único nodo que une a $C$ con $G \setminus C$. Si no es así, es porque hay otro ciclo que conecta a $C$ con $G \setminus C$.

Entonces al eliminar $v$ pueden quedar a lo sumo la cantidad de ciclos simples que une $v$ de componentes conexas, es decir $\frac{d(v)}{2}$.

Por lo tanto, $w(G - v) \leq \frac{d(v)}{2}$

\setcounter{subsection}{6}
\subsection{}

\subsubsection{}

$\Longrightarrow$ )

Suponemos $G$ arbitrariamente atravesable desde $v_o$ y que $\exists C$ circuito simple que no contiene a $v_0$. Considero $G \setminus C$ (saco las aristas de $C$), donde todos los vértices tienen grado par y usando que $G$ euleriano es equivalente a $\exists$ partición de $E$ en circuitos simples, encuentro una partición en ciclos de cada componente conexa y obtengo una partición en ciclos simples de $E$ que contiene a $C$.

Recorro las aristas de todos los ciclos de la partición que contiene a $v_0$, hasta terminar en $v_0$ y no quedan ,ás aristas adyacentes a $v_0$ por recorrer. Como me falta recorrer las aristas de $C$, este circuito no es euleriano. Absurdo. \\

$\Longleftarrow$ )

Veamos que el procedimiento siempre nos da un circuito euleriano:

$G$ euleriano $\Longrightarrow$ todo vértice tiene grado par.

Al ir construyendo el circuito del procedimiento se va formando un camino en el que eventualmente se forma un ciclo $C_1$ (sino habría un vértices de grado impar) que termina en $v_0$ (sino habría un ciclo que no contiene a $v_0$).

Repitiendo el razonamiento, se forman ciclos simples disjuntos $C_1, ..., C_k$ que contienen a $v_0$ (y no hay más cuando termina el procedimiento).

Igual que antes, puedo contruir una partición de ciclos simples que contengan a $C_1$ hasta $C_k$.

Si el procedimiento no genera un circuito euleriano, $\exists$ una arista $e$ que no recorrí que pertenece a algún circuito de la partición $neq$ de los $c_i$. Absurdo porque ese circuito no contiene a $v_0$.

\subsubsection{}

Sea $C = \{C_1, ..., C_k\}$ una partición en ciclos simples de $E$. Cada ciclo aporta como mucho 2 al grado de $v$, cualquiera sea el $v \in V$ y todas las aristas están en algun ciclo $\Longrightarrow d(v) \leq 2k = d(v_0)$ ya que todo ciclo contiene a $v_0$ y son disjuntos.

\subsection{}

\subsubsection{}
Para valores impares.

\subsubsection{}
Si. $K_2$.

\subsection{}

\subsubsection{}

$\Longrightarrow$ )

Sea $G$ un digrafo conexo que tiene un circuito euleriano, es decir que tiene un circuito que recorre todas las aristas sin pasar más de una vez por cada ninguna. 

Como cada vértice $v \in G$ pertenece al circuito euleriano, por lo tanto si comienzo a recorrer el circuito partiendo desde $v$, por cada salida que tenga debe tener obligatoriamente una entrada, para poder formar el ciclo, y de la misma forma. Si no parto de $v$, por cada entrada que tenga tiene que tener una salida, para garantizar que se llegue al nodo desde donde se partió y formar el ciclo. \\

$\Longleftarrow$ )

Sea $G$ un digrafo conexo que cumple que $d_{in}(v) = d_{out}(v)$ $\forall$ vértice $v$. 

\subsection{}

Si un digrafo conexo tiene un circuito euleriano, significa que existe un ciclo dirigido que pasa por todas las aristas. Si un ciclo pasa por todas las aristas y el grafo es conexo, entonces pasa por todos los nodos. Tener un ciclo dirigido que pase por todos los nodos implica que se puede ir desde cualquier nodo a cualquier otro, siguiendo el ciclo dirigido. Lo que es lo mismo que decir que el grafo es fuertemente conexo.