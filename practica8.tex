\section{Práctica 8}

\setcounter{subsection}{3}
\subsection{}

$\Longrightarrow$ )

$G$ tiene un circuito euleriano, osea que si parto de un nodo $v_0$ puedo volver al mismo pasando por todas las aristas, sin repetir alguna. A su vez, todos los nodos de $G$ tienen grado par.

Puede ocurrir que $G$ sea un circuito simple, en cuyo caso tomo $G$ como la única partición, o puede que no. En ese caso puedo formar un circuito simple $C_1$ con un conjunto de aristas de $G$, ya que sé que al haber un circuito hay al menos un circuito simple.

Las aristas de $C_1$ forman un conjunto de la partición que busco. Sea $G_1 = G \setminus C_1$. Todos los vértices de $G_1$ siguen teniendo grado par. Si $G_1$ no tiene aristas ya tenemos la partición que buscábamos. Sino, seguro $G_1$ tiene un circuito simple $C_2$, por ser euleriano. Así iteramos hasta quedarnos sin aristas y nos queda la partición $\{C_1, ..., C_k\}$. \\

$\Longleftarrow$ )

$G$ puede particionarse en circuitos simples $\{C_1, ..., C_k\}$. Considero $G_0$ el grafo con los mismos nodos que $G$ pero sin aristas, sólo nodos aislados. Todos tienen grado 0, osea que todos son de grado par. Ahora agrego $C_1$. Como $C_1$ es ciclo simple, todos sus nodos tienen grado 2, por lo que agregar $C_1$ a $G_0$ forma un grafo $G_1$ en donde cada nodo aumenta de grado en 2 o permanece igual, por lo que son pares. Itero hasta agregar todos los circuitos.

El grafo que se forma es $G$, es conexo y todos sus nodos son de grado par, por lo que $G$ es un grafo euleriano.

\subsection{}

$G$ es un grafo conexo con todos sus vértices de grado par, por lo tanto es un grafo euleriano. Entonces $G$ se puede particionar en ciclos simples.

Para cualquier vértice $v \in G$ se sabe que todas sus aristas pertenecen a algún ciclo simple. Por cada arista de $v$ hay otra que también sale de $v$ que pertenece al mismo ciclo simple. Es decir que $v$ conecta $\frac{d(v)}{2}$ ciclos simples, ya que puedo elegir pares de aristas de $v$ que pertenecen al mismo ciclo.

Al sacar uno de esos pares de aristas, es posible que se forme otra componente conexa porque ese par pertenece a un ciclo simple $C$, lo que hace que se forme un camino simple ya que las aristas que elimino son adyacentes. Para formarse otra componente conexa debe ser $v$ el único nodo que une a $C$ con $G \setminus C$. Si no es así, es porque hay otro ciclo que conecta a $C$ con $G \setminus C$.

Entonces al eliminar $v$ pueden quedar a lo sumo la cantidad de ciclos simples que une $v$ de componentes conexas, es decir $\frac{d(v)}{2}$.

Por lo tanto, $w(G - v) \leq \frac{d(v)}{2}$

\setcounter{subsection}{6}
\subsection{}

\subsubsection{}

$\Longrightarrow$ )

Suponemos $G$ arbitrariamente atravesable desde $v_o$ y que $\exists C$ circuito simple que no contiene a $v_0$. Considero $G \setminus C$ (saco las aristas de $C$), donde todos los vértices tienen grado par y usando que $G$ euleriano es equivalente a $\exists$ partición de $E$ en circuitos simples, encuentro una partición en ciclos de cada componente conexa y obtengo una partición en ciclos simples de $E$ que contiene a $C$.

Recorro las aristas de todos los ciclos de la partición que contiene a $v_0$, hasta terminar en $v_0$ y no quedan ,ás aristas adyacentes a $v_0$ por recorrer. Como me falta recorrer las aristas de $C$, este circuito no es euleriano. Absurdo. \\

$\Longleftarrow$ )

Veamos que el procedimiento siempre nos da un circuito euleriano:

$G$ euleriano $\Longrightarrow$ todo vértice tiene grado par.

Al ir construyendo el circuito del procedimiento se va formando un camino en el que eventualmente se forma un ciclo $C_1$ (sino habría un vértices de grado impar) que termina en $v_0$ (sino habría un ciclo que no contiene a $v_0$).

Repitiendo el razonamiento, se forman ciclos simples disjuntos $C_1, ..., C_k$ que contienen a $v_0$ (y no hay más cuando termina el procedimiento).

Igual que antes, puedo contruir una partición de ciclos simples que contengan a $C_1$ hasta $C_k$.

Si el procedimiento no genera un circuito euleriano, $\exists$ una arista $e$ que no recorrí que pertenece a algún circuito de la partición $neq$ de los $c_i$. Absurdo porque ese circuito no contiene a $v_0$.

\subsubsection{}

Sea $C = \{C_1, ..., C_k\}$ una partición en ciclos simples de $E$. Cada ciclo aporta como mucho 2 al grado de $v$, cualquiera sea el $v \in V$ y todas las aristas están en algun ciclo $\Longrightarrow d(v) \leq 2k = d(v_0)$ ya que todo ciclo contiene a $v_0$ y son disjuntos.

\subsection{}

\subsubsection{}
Para valores impares.

\subsubsection{}
Si. $K_2$.

\subsection{}

\subsubsection{}

$\Longrightarrow$ )

Sea $G$ un digrafo conexo que tiene un circuito euleriano, es decir que tiene un circuito que recorre todas las aristas sin pasar más de una vez por cada ninguna. 

Como cada vértice $v \in G$ pertenece al circuito euleriano, por lo tanto si comienzo a recorrer el circuito partiendo desde $v$, por cada salida que tenga debe tener obligatoriamente una entrada, para poder formar el ciclo, y de la misma forma. Si no parto de $v$, por cada entrada que tenga tiene que tener una salida, para garantizar que se llegue al nodo desde donde se partió y formar el ciclo. \\

$\Longleftarrow$ )

Sea $G$ un digrafo conexo que cumple que $d_{in}(v) = d_{out}(v)$ $\forall$ vértice $v$. 

\subsection{}

\begin{lema}
    Todo digrafo conexo que tiene un circuito euleriano orientado es fuertemente conexo.
\end{lema}

\begin{proof}
    Si un digrafo conexo tiene un circuito euleriano, significa que existe un ciclo dirigido que pasa por todas las aristas. Si un ciclo pasa por todas las aristas y el grafo es conexo, entonces pasa por todos los nodos. Tener un ciclo dirigido que pase por todos los nodos implica que se puede ir desde cualquier nodo a cualquier otro, siguiendo el ciclo dirigido. Lo que es lo mismo que decir que el grafo es fuertemente conexo.
\end{proof}

\setcounter{subsection}{11}
\subsection{}
Porque es un grafo bipartito con una cantidad impar de nodos.

\setcounter{subsection}{13}
\subsection{}

\begin{lema}
    Si un digrafo es completamente conexo tiene un camino hamiltoniano orientado.
\end{lema}

\begin{proof}
    Tomo un nodo cualquiera. Si es el único ya se que cumple. Sino, tomo otro nodo cualquiera. Considero el arco que los une como el camino hamiltoniano $C$. Tomo otro nodo $n$. Lo que voy a hacer es insertarlo en el camino hamiltoniano. Pueden darse varias posibilidades:

    Ahora recorro $C$ buscando un nodo que contenga un arco desde $n$. Sean $v_1, ..., v_k$ los nodos de $C$ siendo $v_1$ el principio y $v_k$ el final.

    \begin{itemize}
        \item Si hay un arco desde $n$ a $v_1$, en este caso $n$ es el nuevo principio. Los uno por medio de ese arco.
        \item Si hay un arco desde $v_k$ hacia $n$, en este caso $n$ es el nuevo final de $C$. Los uno por medio de ese arco.
        \item De lo contrario, recorro los nodos hasta que en $v_i$ encuentre un arco que llegue desde $n$. Una vez que lo encuentro (observemos que este arco debe existir necesariamente ya que si no hubieran arcos que salgan de $n$ entonces todos llegan a $n$ por lo que sería el nuevo final del camino hamiltoniano), inserto a $n$ en el camino. Es decir, elimino el arco que $(v_{i-1}, v_i)$ y agrego $(v_{i-1}, n)$ y $(n, v_i)$. Sabemos que existe $(v_{i-1}, n)$ porque si no, existiría $n, v_{i-1}$ por ser completamente conexo, y por lo tanto hubiésemos elegido ese nodo para hacer la inserción.
    \end{itemize}
\end{proof}

\subsection{}

\begin{lema}
    Un grafo bipartito con un número impar de vértices no contiene un camino hamiltoniano.
\end{lema}

\begin{proof}
    Supongo que tengo un grafo bipartito completo. Comienzo a formar el camino desde cualquier nodo $n_0$. Cada nodo que avance en el camino hace que necesariamente alterne de partición. Como tengo una cantidad impar de nodos hay una de las particiones que tiene mayor cantidad de nodos que la otra. Llamemos $P$ a la mayor y $p$ a la menor.

    Llegará un momento en el que me haya recorrido todos los nodos en $p$ y por lo tanto no pueda continuar mi camino desde $n_i \in P$.

    \begin{itemize}
        \item Si $n_0 \in P$ entonces no existe una arista entre $n_0$ y $n_i$ porque ambos pertenecen a $P$ y por lo tanto no puedo cerrar el circuito.
        \item Si $n_0 \in p$ entonces habré tomado igual cantidad de nodos de $p$ que de $P$. Pero $\#P > \#p$ entonces aún quedan nodos de $P$ por recorrer. No puedo recorrerlos ya que no existen arcos entre $n_i$ y ninguno de ellos porque todos pertenecen a $P$ y por lo tanto no puedo cerrar el circuito.
    \end{itemize}
\end{proof}

\subsection{}

Si $G$ es un grafo con $n \geq 4$ y $d$ (grado mínimo) $\geq n - 2$ entonces $G$ tiene un circuito hamiltoniano, ya que como $n \geq 3$ y $d(v) \geq \frac{n}{2}$, $G$ cumple las condiciones del Teorema de Dirac. 