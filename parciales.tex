\section{Parciales}

\subsection*{21/12/2013}

\begin{enumerate}

\item
Un grafo se dice planar maximal si y sólo si es simple y planar pero deja de
serlo si se le agrega cualquier eje entre vértices existentes.

\begin{enumerate}[label=\alph*)]
\item
Demostrar que si $G$ es planar maximal entonces es conexo.

~

Supongo que $G$ es planar maximal pero no es conexo. Esto significa que $G$
tiene 2 o más componentes conexas $\{C_1, ..., C_k\}$. Si agrego una arista
$(a, b)$ con $a \in C_i, b \in C_j$, siendo $a$ y $b$ vértices alcanzables
por la región externa, entonces $G$ sigue siendo planar. Absurdo.

\item
Exhibir todos los grafos planares maximales no isomorfos de $n \leq 5$.
Justificar.

~

Para $n < 5$ no pueden formarse grafos no planares ya que no hay manera de
que un grafo con $n < 5$ nodos sea homeomorfo a $K_5$ o a $K_{3,3}$, ya que
cualquier contracción de aristas que haga decrementa en 1 la cantidad de
nodos del grafo. Por lo tanto no hay grafos planares maximales con $n < 5$.

Para $n = 5$, el único grafo planar maximal es $K_5$ sin una arista. Esto es
porque teniendo $n = 5$, la única forma de que al agregarle una arista deje
de ser planar, es que sea homeomorfo a $K_5$ ó a $K_{3,3}$. A $K_{3,3}$ no
puede ser por que tiene menor cantidad de nodos, lo cual nos deja a $K_5$ como
única opción. Para ser homeomorfo a $K_5$, debo poder hacer contracciones de
aristas hasta poder formarlo, pero al hacerlo disminuiría la cantidad de
nodos, por lo que la única contracción posible es la nula, es decir, no
contraer ninguna arista. Luego, sólo puede ser homeomorfo a $K_5$ si al
agregar una arista, el grafo es $K_5$.
\end{enumerate}

\item
Sea $G$ un grafo que tiene dos vértices $u \not= v$ tales que todo vértice
adyacente a $u$ es también adyacente a $v$. Demostrar que
$\chi(G) = \chi(G - u)$. Notar que $u$ y $v$ no son adyacentes porque si lo
fueran $v$ sería adyacente a sí mismo.

~

Supongo que $\chi(G) > \chi(G - u)$. Tomo $G - u$ y le agrego $u$ con las
aristas correspondientes para formar $G$. Para que haya aumentado el valor
cromático de $G - u$ al agregar $u$ es necesario que se cada vecino de $u$
requiera un color distinto así al agregar $u$, le asigno un nuevo color.
Pero eso no es posible ya que $v$ tiene los mismo vecinos. Si $u$ necesitase
un nuevo color, entonces $v$ no tendría asignado un color válido. Esto prueba
que $\chi(G) \leq \chi(G - u)$.

Como $G - u$ es un subgrafo de $G$, se cumple que $\chi(G) \geq \chi(G - u)$.

Por lo tanto $\chi(G) = \chi(G - u)$.

\item
Demostrar que el siguiente problema es NP-completo.

CLIQUE MÁS PESADA

Entrada: un grafo $G$ en el cual cada vértice tiene asociado un peso entero
positivo, y un entero positivo $k$.

Pregunta: ¿tiene $G$ un subgrafo completo de peso total $k$ o mayor?

~

Primero demostraremos que el problema de la clique más pesada, que llamaremos
$\pi$, pertenece a la clase de complejidad NP. pendiente...

Ahora consideremos el problema de clique máxima, que llamaremos $\pi'$, que
sabemos que es NP-completo, para así formar una reduccíon polinomial
$f : D_{\pi'} \longrightarrow D_\pi$ de $\pi'$ a $\pi$. De esta forma
demostraremos que $\pi \in$ NP-hard.

CLIQUE MÁXIMA

Entrada: un grafo $G$ y un entero positivo $k$.

Pregunta: ¿Tiene $G$ un subgrafo completo de tamaño $k$?

Definimos $f : D_{\pi'} \longrightarrow D_\pi$ tal que $f(G, k) = (G', k)$
donde $G'$ es una copia de $G$ en donde se le asigna un peso de 1 a cada
vértice.

\begin{enumerate}
\item
$f$ mapea instancias de $\pi'$ en $\pi$.

Las instancias de $\pi'$ son pares que incluyen un grafos arbitrario y un
número entero positivo $k$. $f(G, k) = (G', k)$ es una instancia de $\pi$ ya
que $k$ es un número entero positivo como $\pi$ requiere que sea. Y en cuanto
a $G'$, debe ser un grafo cualquiera que tenga pesos enteros positivos en los
vértices. Esto tambien se cumple.

\item
$f$ es polinomial, es decir, hay un algoritmo que computa $f$ en tiempo
polinomial en el tamaño de la entrada.

\textbf{Algoritmo}:

\begin{enumerate}
\item Copiar $G$
\item Asignar 1 en el peso de cada vértice
\item Copiar $k$
\end{enumerate}

El tamaño de la entrada es $S = \Omega(n^2)$ y el algoritmo toma tiempo
$\ord(S)$ por lo que $f$ es polinomial.

\item
$\forall I \in D_{\pi'}$, $I \in Y_{\pi'} \Longleftrightarrow f(I) \in Y_\pi$.

Si $I \in Y_{\pi'} \Longrightarrow G$ tiene un subgrafo completo de tamaño $k$

$\Longrightarrow f(I)$ tiene un subgrafo completo de tamaño $k$ con peso 1 en
cada uno de sus vértices.

$\Longrightarrow f(I)$ tiene $G$ un subgrafo completo de peso total $k$ o
mayor

$\Longrightarrow f(I) \in Y_\pi$

Esto demuestra la ida, para la vuelta es necesario observar que $f(I)$ es
una instancia particular, ya que todos los vértices de $G'$ tiene peso 1.


\end{enumerate}
\end{enumerate}