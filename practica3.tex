\section{Práctica 3}

\setcounter{subsection}{1}
\subsection{}

\begin{codesnippet}
\begin{verbatim}
hanoi(int discos)

    si discos == 1 entonces

        moverUnicoDisco()

    sino
    
        // Mueve la torre de discos - 1 discos a otro lugar, dependiendo de la paridad
        hanoi(discos - 1)

        // Mueve dependiendo de la paridad
        moverUltimoDisco()

        // Mueve la torre de discos - 1 discos sobre la base ya movida
        hanoi(discos - 1)

    fin si

fin hanoi
\end{verbatim}
\end{codesnippet}

\setcounter{subsection}{3}
\subsection{}

\begin{codesnippet}
\begin{verbatim}
torneo(lista)

    para i entre 0 y n - 1
        para j entre 0 y n / 2
            fixture[i].agregar(lista[j],lista[n/2 + j])
        fin para
    fin para

    torneo(lista[0..n/2))
    torneo(lista[n/2..n))
fin torneo    
\end{verbatim}
\end{codesnippet}

\setcounter{subsection}{6}
\subsection{}
\subsubsection{}

\begin{codesnippet}
\begin{verbatim}
coeficientesBinomiales(int grado)

    vector<int> coefs
    coefs.agregar(1)

    para i entre 0 y grado
        coefs.agregar(1)
        para j entre 1 y coefs.largo() - 1
            coefs[j] = coefs[j] + coefs[j+1]
        fin para
    fin para

    devolver coefs

fin coeficientesBinomiales
\end{verbatim}
\end{codesnippet}

\setcounter{subsection}{8}
\subsection{}

\subsubsection{}

\begin{codesnippet}
\begin{verbatim}

arreglo[v.tamaño()]<mapa<int,(bool esPosible,lista<char> opList)>> ops

operaciones(lista<int> v, int w)

    // Chequea si ya fue calculada una manera de formar w. De ser asi termina, sino la calcula
    // y la guarda
    si !ops[v.tamaño() - 1].existe(w) entonces

        si v.tamaño() == 1 entonces
            ops[0].agregar((v.primero() == w, []))
        sino
            
            operaciones(v.sinElUltimo(), w - v.ultimo())

            (bool, lista<char>) info = ops[v.tamaño() - 2].obtener(w - v.ultimo())
            si info.esPosible entonces
               ops[v.tamaño() - 1].agregar(w, (true, info.opList + '+'))
            sino
                ops[v.tamaño() - 1].agregar(w, (false, [])
            fin si

            operaciones(v.sinElUltimo(), w / v.ultimo())

            (bool, lista<char>) info = ops[v.tamaño() - 2].obtener(w / v.ultimo())
            si info.esPosible entonces
               ops[v.tamaño() - 1].agregar(w, (true, info.opList + '*'))
            sino
                ops[v.tamaño() - 1].agregar(w, (false, [])
            fin si

            operaciones(v.sinElUltimo(), raiz(v.ultimo(), w))

            (bool, lista<char>) info = ops[v.tamaño() - 2].obtener(raiz(v.ultimo(), w))
            si info.esPosible entonces
                ops[v.tamaño() - 1].agregar(w, (true, info.opList + '^'))
            sino
                ops[v.tamaño() - 1].agregar(w, (false, [])
            fin si
            
        fin si
    fin si
fin operaciones
\end{verbatim}
\end{codesnippet}

\subsubsection{}
La complejidad temporal es $\ord(n)$ con $n$ el largo de $v$. Esto es porque para cada elemento se calcula cada una de las 3 operaciones con el siguiente elemento, una unica vez. No se produce una ramificación exponencial debido a que las soluciones parciales se almacenan.

\setcounter{subsection}{10}
\subsection{}

\subsubsection{}
No. La más eficiente es insertar una c entre las dos b y borrar la segunda a

\subsubsection{}

\begin{codesnippet}
\begin{verbatim}

arreglo[v.tamaño()]<mapa<string palabra,int cambios>> cambios

int transformar(String u, String v, int indice)

    si !cambios[indice].existe(v) entonces

        si u.primero() == v.primero() entonces

            sinCam = transformar(u.cola(), v.cola(), indice + 1)
            cambios[indice].agregar(v, sinCam)

        sino
    
            // Borrando
            borr = transformar(u.cola(), v, indice)

            //Agregando
            agre = transformar(u, v.cola(), indice + 1)

            //Cambiando
            cam = transformar(u.cola(), v.cola(), indice + 1)

            cambios[indice].agregar(v, min(borr, agre, cam) + 1)

        fin si

    fin si

    devolver cambios[indice].obtener(v)

fin transformar

devolver transformar(u, v, 0)

\end{verbatim}
\end{codesnippet}

\setcounter{subsection}{12}
\subsection{}

\subsubsection{}
Habría que considerar $64 \choose 8$ posibilidades.

\subsubsection{}

\begin{codesnippet}
\begin{verbatim}
lista<int> reinas(int cantReinas, lista<int> posiciones)
    si cantReinas == 0 entonces
        devolver posiciones
    sino
        para i entre 0 y 64
            si !colisiona(i, posiciones) entonces
                lista<int> posisionesRec = reinas(cantReinas - 1, posiciones + i)
                si posicionesRec.largo() == cantReinas entonces
                    devolver posicionesRec
                fin si
            fin si
        fin para

        devolver posiciones

    fin si
fin reinas
\end{verbatim}
\end{codesnippet}

\subsection{}

\begin{codesnippet}
\begin{verbatim}
int mochila(vector<(int w, int v)> objetos, int W)

    int max = 0

    para i entre 0 y objetos.largo()
        int cantObjetoI = W / objetos[i].w
        int mochilaRec = mochila(objetos.sin(i), W - cantObjetoI)
        max = maximo(max, mochilaRec) + cantObjetoI * objetos[i].v
    fin para

    devolver max
fin mochila
\end{verbatim}
\end{codesnippet}

\subsection{}

\setcounter{subsubsection}{1}
\subsubsection{}
La cantidad de subconjuntos de $S$ es igual a $partes(S)$ por lo que su complejidad es de $\ord(2^n)$