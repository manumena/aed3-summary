\section{Práctica 7}

\setcounter{subsection}{1}
\subsection{}

\subsubsection{}
(A,B,1)(A,C,2)(C,B,-999)

Si Dijkstra empieza desde $A$ nunca llega a la arista (B,C) ya que hace el siguiente recorrido: (A,B)(A,C) cuando el mejor camino sería (A,C)(C,B)

\subsubsection{}
Consiste en moverse entre los nodos yendo siempre al que menor camino total represente.

\subsubsection{}
Si ya que toma la mejor decisión, osea el mínimo camino, en cada iteración.

\setcounter{subsection}{4}
\subsection{}

\subsubsection{}
Se relajan todas las aristas una última vez. Si hubo algún cambio en las distancias, entonces hay un ciclo negativo.

Si no todos los vértices son alcanzables se puede correr el algoritmo cuantas veces sea necesario para cada nodo con lognitud $\infty$. Para evitar que la complejidad del algoritmo incremente es necesario que el bucle principal finalice si es que no se detectan cambios en cuanto a la iteración anterior. De esta forma se evita repetir la relajacíon de aristas $n^2$ veces.