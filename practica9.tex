\section{Práctica 9}

\setcounter{subsection}{1}
\subsection{}



\subsection{}

\subsubsection{}
$r = m - n + c + 1$ donde $c$ es el número de componentes conexas

\subsubsection{}
Sea un grafo planar no conexo $G_N$, puedo construir un grafo planar conexo $G_C$ uniendo nodos de componentes conexas distitnas por medio de aristas. Llamemos $m_a$ a esta cantidad de aristas agregadas.

Sea $m$ la cantidad de aristas de $G_C$ y $n$ la cantidad de nodos, se cumple que $m \leq 3n - 6$. Ahora, $G_N$ tiene la misma cantidad de nodos pero $m_a$ aristas menos, es decir, $m - m_a$ aristas. Por lo tanto, $m - m_a \leq m \leq 3n - 6$.

\subsection{}

\begin{lema}
    Todo grafo planar donde todo vértice tiene grado mayor o igual a 3, se verifica que $m \leq 3r - 6$.
\end{lema}

\begin{proof}
    ~

    $3r - 6 = 3(m - n + 2) - 6 = 3m - 3n$

    Quiero ver que $m \leq 3m - 3n$

    Lo que es lo mismo que $\frac{3n}{2} \leq m$

    Como cada vértice tiene grado mayor o igual a 3, $m = \frac{\sum_{i=1}^{n}d(v_i)}{2} \geq \frac{\sum_{i=1}^{n}3}{2} = \frac{3n}{2}$.
\end{proof}

\subsection{}

\subsubsection{}
Supongo un grafo $G = (V, E)$ donde todos sus vértices tienen grado mayor o igual a 6. Como mínimo debe tener 7 nodos, todos conectados entre sí. Tomamos $G = K_7$ por ser el de menor cantidad de nodos y aristas.

Este grafo no es planar ya que no cumple con el lema \ref{planarMayor3}.

Agrego un nodo $n$ a $V$, de manera que agregue la menor cantidad de aristas posibles. Para esto voy a elegir 6 nodos $v_a, v_b, v_c, v_d, v_e, v_f \in V$ donde $(v_a, v_b), (v_c, v_d), (v_e, v_f) \in E$. Agrego al grafo las aristas $(v_a, n), (v_b, n), (v_c, n), (v_d, n), (v_e, n), (v_f, n)$ y puedo sacar $(v_a, v_b), (v_c, v_d), (v_e, v_f)$ de $E$ ya que al hacerlo cada nodo sigue vuelve a tener grado 6.

El numero de nodos aumenta en 1 y el de aristas aristas en 3. Esta es la cantidad mínima de aristas que puedo agregar por cada nodo. En principio $G$ cumplía que $m = \frac{7 \times 6}{2} = 21 = 3n$, entonces si por cada nodo que agrego la cantidad de aristas incrementa como mínimo en 3, entonces $m \geq 3n$ pero entonces $m \not \leq 3n - 6$. Lo que significa que $G$ no puede ser planar si todos sus vértices tienen grado mayor o igual a 6.

Por lo tanto un grafo planar tiene al menos un vértice de grado menor o igual a 5.

\subsubsection{}
Supongo un grafo $G$ planar donde todo vértice tiene grado mayor o igual a 5.

Por lema \ref{aristasGrados} se que $m = \frac{\sum_{i = 1}^{n}d(v_i)}{2} \geq \frac{5n}{2}$.

Y por lema \ref{planarMayor3} se que $m \leq 3n - 6$.

Entonces $\frac{5n}{2} \leq 6n - 6$, lo que es lo mismo que $n \geq 12$. Por lo tanto $G$ tiene mas de 11 vértices.

----------------------------------

Supongo un grafo $G = (V, E)$ donde todos sus vértices tienen grado mayor o igual a 5. Como mínimo debe tener 6 nodos, todos conectados entre sí. Tomamos $G = K_6$ por ser el de menor cantidad de nodos y aristas.

Este grafo no es planar ya que no cumple con el lema \ref{planarMayor3}.

Agrego un nodo $n$ a $V$, de manera que agregue la menor cantidad de aristas posibles. Para esto voy a elegir 5 nodos $v_a, v_b, v_c, v_d, v_e \in V$ donde $(v_a, v_b), (v_c, v_d) \in E$. Agrego al grafo las aristas $(v_a, n), (v_b, n), (v_c, n), (v_d, n), (v_e, n)$ y puedo sacar $(v_a, v_b), (v_c, v_d)$ de $E$ ya que al hacerlo cada nodo sigue vuelve a tener grado 5, con excepción de $v_e$ que tiene grado 6.

El numero de nodos aumenta en 1 y el de aristas aristas en 3. Ahora puedo repetir el proceso anterior, pero esta vez no voy a elegir a $v_e$. Voy a elegir otros 5 nodos $v_a', v_b', v_c', v_d', v_e' \in V$ donde ninguno sea $v_e$ y donde $(v_a', v_b'), (v_c', v_d'), (v_e', v_e) \in E$. Luego agrego $(v_a', n), (v_b', n), (v_c', n), (v_d', n), (v_e', n)$ y saco $(v_a', v_b'), (v_c', v_d'), (v_e', v_e)$. En esta iteración pude sacar 3 aristas y agregar 5, por lo que se agregar 2 efectivamente. Esta es la manera de agregar la cantidad mínima de aristas por nodo.

De esta forma puedo formar un grafo que cumpla que todos sus vértices tienen grado mayor o igual a 5 con la menor cantidad de aristas posibles para cada $n$.

En principio $G$ cumplía que $m = \frac{6 \times 5}{2} = 15 = 3n$, entonces si por cada 2 nodos que agrego la cantidad de aristas incrementa en 5

\subsubsection{}
Sea $G$ un grafo con 11 o más vértices que verifica que $m \leq 3n - 6$, es decir es planar. Quiero ver que $\overline{G}$ no es planar.

Sea $\overline{m}$ la cantidad de aristas de $\overline{G}$, sabemos que $\overline{m} = \frac{n(n - 1)}{2} - m$, y por lo tanto $\overline{m} \geq \frac{n(n - 1)}{2} - 3n + 6$.

Para que $\overline{G}$ sea planar debe cumplirse que 

\begin{center}
    $\frac{n(n - 1)}{2} - 3n + 6 \leq \overline{m} \leq 3n - 6$



\end{center}

\subsection{}
No. Pero sí es cierto que un grafo es planar $\Longleftrightarrow$ no contiene ningún subgrafo homeomorfo a $K_{33}$ ó $K_5$.